%% multiple-formats.tex
%% Copyright 2024 Gareth Walker
%
% This work may be distributed and/or modified under the
% conditions of the LaTeX Project Public License, either version 1.3
% of this license or (at your option) any later version.
%
% This work has the LPPL maintenance status `maintained'.
% 
% The Current Maintainer of this work is Gareth Walker.

\documentclass[a4paper,12pt]{article}
\usepackage[margin=2.5cm]{geometry}
\usepackage{graphicx}
\usepackage{float}
\usepackage{metalogo}
\usepackage{hyperref}

\begin{document}
\title{Multiple output formats from the same \LaTeX\ source\thanks{The
    information in this document is offered in good faith and I hope
    that others find it helpful.  Much of it has come, directly or
    indirectly, from the author of the make4ht tool, Michal
    Hoftich. I've taken math snippets from
    \href{https://euclid.colorado.edu/~hilljb/latex/intro.to.latex.tex}{this
      intro to writing math with \LaTeX}.  There are absolutely no
    guarantees as to whether, and how well, these tools will work in
    your case. I am not the author of any of the tools referred to in
    this document.  Corrections and improvements to this document are
    welcome.}}  \author{Gareth Walker}
\maketitle
\tableofcontents

\section{Introduction}

There are times when it can be useful to use the same \LaTeX\ source
code and produce multiple output formats.  One such case is where you
are preparing materials for students and you want them to have, for
example, a PDF `reference' version but would also like them to have
the same materials in HTML format and in formats they can use in
LibreOffice (ODT) and in Microsoft Word (DOCX).  This document gives
you some tips on how to achieve that. If you are currently looking at
an output file in PDF, HTML, ODT, or DOCX format \textbf{these all
  originated from the same single \LaTeX\ source file}. So as well as
documenting the process of conversion, this file functions as an
example of the output which can be produced.

\section{The main tool: make4ht}

The main tool in the conversion process described here is Michal
Hoftich's excellent \href{https://ctan.org/pkg/make4ht}{make4ht} tool
(and the related \href{https://ctan.org/pkg/tex4ht}{tex4ht} package).
So, install those things if you don't already have them. There is
\href{https://www.kodymirus.cz/tex4ht-doc/tex4ht-doc.html}{quite
  extensive documentation} if you need it, but this document should
help you get started.

\subsection{Getting HTML output}

This is as simple as doing this from the command line:
\begin{verbatim}
make4ht multiple-formats.tex
\end{verbatim}

\subsection{Getting LibreOffice (ODT) output}

You can do this from the command line:
\begin{verbatim}
make4ht -f odt multiple-formats.tex
\end{verbatim}

\subsection{Getting Microsoft Word (DOCX) output}
\label{msword}
Once you have the ODT file, you can open the file in LibreOffice and
use ``Save as...'' to save the file in DOCX format.  If you are able
to call LibreOffice from the command line (terminal) on your system,
you may be able to do the conversion that way, something like this:
\begin{verbatim}
libreoffice7.5 --convert-to docx multiple-formats.odt
\end{verbatim}

\subsection{Putting it all together}

Depending on your system, you may be able to put all these steps
together in a shell script so that running that shell script on a
file generates some or all of the output formats (PDF, HTML, ODT,
DOCX). On my system (Ubuntu Linux) I can have a bash script like this:
\begin{verbatim}
#!/bin/bash
fileName=$(basename "$1" | cut -d. -f1)
latexmk $fileName
make4ht $fileName
make4ht -f odt $fileName
libreoffice7.5 --convert-to docx $fileName.odt
\end{verbatim}
If I save this as \verb!makeMultipleFormats!, make that file
executable, and possibly put it somewhere in my system's path, I can
then run this command:
\begin{verbatim}
makeMultipleFormats multiple-formats.tex
\end{verbatim}
That single command gets me PDF, HTML, ODT and DOCX versions of the
same \LaTeX\ source, all at the same time. Nice!

\section{Other things}

This section contains some advice on some of the situations which
might crop up, and some suggestions of what to do.

\subsection{Graphics}

You can include graphics like this, optionally specifying a size and
alt-text to describe the image:
\begin{verbatim}
\includegraphics[alt={a colour picture of a tiger},width=3cm]{tiger}  
\end{verbatim}
\includegraphics[alt={a colour picture of a tiger},width=3cm]{tiger}\\
The alt-text will be preserved in the HTML and ODT versions, but seems
to be lost when converting to DOCX from LibreOffice.

To get the image size you want in the ODT and DOCX file, you may need
to do a bit of work. One thing you can do is use the \verb!ebb!
command from the command line to create a special file which contains
the dimensions of the graphics, e.g.
\begin{verbatim}
  ebb -x tiger.pdf
\end{verbatim}
(This works on other types of graphics as well, not just PDF.)  There
is more detail on this on
\href{https://tex.stackexchange.com/a/568853/110494}{StackExchange}.

\subsection{Math}

Numbered equations such as
\begin{verbatim}
\begin{equation}
\sum_{k=1}^5(k^2+2) = (1^2+2)+(2^2+2)+(3^2+2)+(4^2+2)+(5^2+2)
\end{equation}
\end{verbatim}
can be converted to graphics:
\begin{equation}
\sum_{k=1}^5(k^2+2) = (1^2+2)+(2^2+2)+(3^2+2)+(4^2+2)+(5^2+2)
\end{equation}
The graphics will have alt-text generated automatically.

Inline math (e.g. \verb!$h(x)=e^x$!) can appear in the text:
$h(x)=e^x$.  There are options to convert math into other accessible formats,
including MathML and MathJax: see the
\href{https://www.kodymirus.cz/tex4ht-doc/BasicTutorial.html#math-options}{documentation}.

\subsection{Tables}

Tables are converted.
\begin{table}[H]
  \centering
  \begin{tabular}{l | l l}
    a & b & c \\\hline
    1 & 2 & 3 \\
    2 & 3 & 4
  \end{tabular}
  \caption{A table}
\end{table}

\subsection{Phonetic symbols}

Phonetic symbols seem to convert fine when done something like this:
\begin{verbatim}
\documentclass{article}
\usepackage{fontspec}
\newfontfamily\ipafont{Doulos SIL}
\newcommand\ipa[1]{{\ipafont #1}}
\begin{document}
This is a test
\ipa{[(input unicode IPA here)]}
This is a test
\end{document}
\end{verbatim}
Since you need \XeLaTeX\ or \LuaTeX\ to use the \verb!fontspec!
package, to do the conversion to conversion you need e.g.
\begin{verbatim}
make4ht -x file.tex
\end{verbatim}
and for ODT output e.g.
\begin{verbatim}
make4ht -x -f odt file.tex
\end{verbatim}
A DOCX file can be created from the ODT file using LibreOffice in the
same way as described in section~\ref{msword}.

\subsection{Syntax trees}

This is a document showing a syntax tree using the qtree package:
\begin{verbatim}
\documentclass{article}
\usepackage{qtree}
\usepackage{ulem}
\begin{document}
This is a tree.
\Tree[.TP\qroof{Valentine}.DP\textit{\begin{scriptsize}i\end{scriptsize}}
[.T' [.T [.V be ] [.T -ed ]] [.VP [.V' [.\sout{V} ][.VP
[.\sout{DP}\textit{\begin{scriptsize}i\end{scriptsize}} ][.V' [.V
writing ] \qroof{a book}.DP ]]]]]]
\end{document}
\end{verbatim}
The conversion to HTML seems robust, e.g.
\begin{verbatim}
make4ht tree.tex "svg"
\end{verbatim}
The ``svg'' switch to get graphics in the SVG format seems to be
required in order to preserve the lines in the graphic representing
branches.

To create a ODT or DOCX version, you can open the HTML file in
LibreOffice, then save it in those other formats.  The size of the
graphics in the final files may need to be adjusted.

\subsection{Heading levels}
 
You can change the style formatting to start at e.g. ``Heading 1'' by
using a configuration file like this:
\begin{verbatim}
\Preamble{xhtml}
\Configure{Heading-2}{Heading 1}
\Configure{Heading-3}{Heading 2}
\Configure{Heading-4}{Heading 3}
\begin{document}
\EndPreamble
\end{verbatim}
If you save the configuration file as \verb!heading.cfg! you can then do e.g.
\begin{verbatim}
make4ht -f odt -c heading.cfg file.tex
\end{verbatim}

\subsection{Creating ZIP files of HTML versions}

There is some information about this on \href{https://tex.stackexchange.com/q/656496/110494}{StackExchange}.

\subsection{Putting HTML versions of documents on Blackboard}

Make a ZIP file of all the files you need to make available (e.g.
the HTML file, CSS file, any graphics).  Upload your ZIP files
to the content collection as a package, so that it unzips once
uploaded. You then have to give all the files the correct
permissions. In the content collection, find the folder you want
to make available.  Click on the `permissions' icon for that
folder. That will take you to a list of who can access the
contents of the folder, and what they can do with the (e.g. read,
write, modify, manage). Click ``Select Specific Users By Place'',
then ``Course''. Choose the course (module) taken by the
students to whom you want to give access, check the box(es),
scroll to the very bottom to ``select roles'' and select e.g.
``Student'' or ``All course users'', check the permissions you
will give (probably just ``Read''), then click ``Submit''.

\end{document}
%%% Local Variables:
%%% mode: latex
%%% TeX-master: t
%%% End:
