%% convtran.tex
%% Copyright 2023 Gareth Walker
%
% This work may be distributed and/or modified under the
% conditions of the LaTeX Project Public License, either version 1.3
% of this license or (at your option) any later version.
% The latest version of this license is in
%   http://www.latex-project.org/lppl.txt
% and version 1.3 or later is part of all distributions of LaTeX
% version 2005/12/01 or later.
%
% This work has the LPPL maintenance status `maintained'.
% 
% The Current Maintainer of this work is Gareth Walker.
%
% This work consists of the files convtran.sty and convtran.tex.

\documentclass[a4paper]{article}
\usepackage[toc]{multitoc}
\renewcommand*{\multicolumntoc}{2}
\setlength{\columnseprule}{0.5pt}
\usepackage{hyperref}
\usepackage[margin=2.5cm]{geometry}
\usepackage{convtran}
\begin{document}
\title{convtran.sty}
\author{Gareth Walker}
\maketitle
\tableofcontents
\setlength{\parskip}{1em}
\setlength{\parindent}{0em}

\section{Introduction}

\verb!convtran.sty! is intended to simplify the presentation of
transcriptions of conversation with \LaTeX.  It does this primarily
through a modified list environment, supplemented by a palette of
commands for the user to adjust the trancriptions according to the
precise circumstances in which the package is being used.

Caveat: this documentation is only intended to give an overview of the
package, and outline its main features; see also the package itself
for comments on the functions.

\section{Some examples}

\subsection{A simple example}
\begin{convtran}
  \turn{Bob:} Hello Jim. \hl
  \turn{Jim:} Hello Bob.
\end{convtran}
\noindent The above example was created with this:
\begin{verbatim}
\begin{convtran}
  \turn{Bob:} Hello Jim. \hl
  \turn{Jim:} Hello Bob.
\end{convtran}
\end{verbatim}

\subsection{A more complicated example}
\begin{convtran}
  \turn{D:} O:h (I k-)=
  \turn{A:} =Dz that make any sense to you?
  \turn{C:} Mn mh. I don’ even know who she is.
  \turn{A:} She’s that’s, the Sister Kerrida, [who,
  \turn{D:} ~~~~~~~~~~~~~~~~~~~~~~~~~~~~~~~~~~[.hhh
  \turn{D:} Oh [\underline{that’s} the one you to:ld [me you bought
  \turn{C:} ~~~[Oh-~~~~~~~~~~~~~~~~~~~~~~[
  \turn{A:} ~~~~~~~~~~~~~~~~~~~~~~~~~~~~~[\underline{Ye:h}
\end{convtran}
This is the code required:
\begin{verbatim}
\begin{convtran}
  \turn{D:} O:h (I k-)=
  \turn{A:} =Dz that make any sense to you?
  \turn{C:} Mn mh. I don’ even know who she is.
  \turn{A:} She’s that’s, the Sister Kerrida, [who,
  \turn{D:} ~~~~~~~~~~~~~~~~~~~~~~~~~~~~~~~~~~[.hhh
  \turn{D:} Oh [\underline{that’s} the one you to:ld [me you bought
  \turn{C:} ~~~[Oh-~~~~~~~~~~~~~~~~~~~~~~[
  \turn{A:} ~~~~~~~~~~~~~~~~~~~~~~~~~~~~~[\underline{Ye:h}
\end{convtran}
\end{verbatim}
Note that the overlapping talk has to be aligned manually using spaces
(\verb!~!) or commands like \verb!\hspace{3em}! or \verb!\hphantom{Oh }!.

\subsection{Numbered examples}
You can use a package like \verb!gb4e! to provide numbered examples (note that \verb!gb4e.sty! has to be loaded after \verb!convtran!):
\begin{verbatim}
(...)
\usepackage{gb4e}
(...)
\begin{exe}
  \ex An example
  \begin{convtran}
    \turn{Bob:} Hello Jim. \hl
    \turn{Jim:} Hello Bob.
  \end{convtran}
\end{exe}
\end{verbatim}

\subsection{Line breaks}
One major benefit of this package, being based on the \verb!list!
environment, is that line breaking is handled automatically.  The
following has only two \verb!turn! commands:
\begin{convtran}
\turn{Bob:} This is a test to see how long I should talk for to get
onto the next line. Seems I have to talk for longer than I expected.
\turn{Jim:} This is also a test of the same thing, Bob my old
mate. I'll continue for a bit longer, just to make sure.
\end{convtran}
\begin{verbatim}
\begin{convtran}
  \turn{Bob:} This is a test to see how long I should talk for to get
  onto the next line. Seems I have to talk for longer than I expected.
  \turn{Jim:} This is also a test of the same thing, Bob my old
  mate. I'll continue for a bit longer, just to make sure.
\end{convtran}
\end{verbatim}

To insert line breaks manually try \verb!\\!:
\begin{convtran}
  \turn{Bob:} This is a test to see how long I should talk\\for to get
  onto the next line. Seems I have\\to talk for longer than I expected.
  \turn{Jim:} This is also a test of the same thing,\\Bob my old
  mate. I'll continue for a bit\\ longer, just to make sure.
\end{convtran}
\begin{verbatim}
\begin{convtran}
  \turn{Bob:} This is a test to see how long I should talk\\for to get
  onto the next line. Seems I have\\to talk for longer than I expected.
  \turn{Jim:} This is also a test of the same thing,\\Bob my old
  mate. I'll continue for a bit\\ longer, just to make sure.
\end{convtran}
\end{verbatim}
Manual line breaks can be useful when aligning talk in overlap.

\subsection{Referring to lines by number}

You can use the \verb!\linelabel! command provided by the \verb!lineno! package, or the shorthand \verb!\lab!. This code:
\begin{verbatim}
\begin{convtran}
  \turn{Bob:} Hello Jim.\linelabel{bob}
  \turn{Jim:} Hello Bob.\lab{jim}
\end{convtran}
First is line~\ref{bob}, followed by line~\ref{jim}.
\end{verbatim}
produces this:
\begin{convtran}
  \turn{Bob:} Hello Jim.\linelabel{bob}
  \turn{Jim:} Hello Bob.\lab{jim}
\end{convtran}
First is line~\ref{bob}, followed by line~\ref{jim}.

\section{Adjusting the typesetting}

There are many aspects of the layout which might need fine-tuning in
order to handle different page layouts, sizes, or even just personal
preference as to the placement of elements.  The elements are designed
to be easily manipulated, and while the package defaults are intended
to be reasonable, more often than not the user will have to specify
aspects of the layout in the preamble.

There are important dimensions which are adjustable, either on a
case by case basis, or in the preamble:
\begin{enumerate}
  \item placement of arrow, with \verb!\setarrowplace{length}!
  \item placement of line numbers, with \verb!\setlinenoplace{length}!
  \item font of main transcription, with \verb!\setconvfont{font}!
  \item font of labelled arrows with \verb!\setarrowfont{font}!
  \item font of gloss line, translation line, or information line with
    \verb!\setglosfont!, \verb!\settranfont!, and \verb!\setinfofont!
    respectively
  \item amount speaker label is indented from the left with
    \verb!\setconvindent{length}!
  \item width of the speaker label with
    \verb!\setconvlabelwidth{length}! in the preamble, or a length in
    square brackets following \verb!\begin{convtran}! to change
    individual cases
\end{enumerate}
  
So if you wanted to adjust some of the parameters for all cases, you might
have a preamble which includes this:
\begin{verbatim}
\usepackage{convtran}
\setlinenoplace{1cm}
\setarrowplace{0cm}
\setconvindent{2cm}
\setconvlabelwidth{3cm}
\setconvfont{\sf\large}
\setarrowfont{\tt\Large}
\end{verbatim}
If you wanted to change some of the parameters for a smaller number of
cases (and therefore didn't want to change the defaults for the entire
document) you might do something like this, which adjusts the
parameters only for that particular case.
\begin{verbatim}
\begin{convtran}[3cm]
  \setlinenoplace{1cm}
  \setarrowplace{0cm}
  \setconvindent{2cm}
  \setconvfont{\sf\large}
  \setarrowfont{\tt\large}
  \turn{Bob:} Hello Jim. \hl
  \turn{Jim:} Hello Bob. \hllab{-0.5cm}{A}
\end{convtran}
\end{verbatim}
Both produce this (which is ugly, but it shows the features):
\begin{convtran}[3cm]
  \setlinenoplace{1cm}
  \setarrowplace{0cm}
  \setconvindent{2cm}
  \setconvfont{\sf}
  \setarrowfont{\tt}
  \turn{Bob:} Hello Jim. \hl
  \turn{Jim:} Hello Bob. \hllab{-0.5cm}{A}
\end{convtran}

You should bear in mind that trial-and-error will almost certainly
have to employed to successfully redefine these lengths for a good
layout.  For instance, you may find that some lengths require positive
values while others require negative ones.  To give a few clues, bear
in mind that 
\begin{itemize}
\item \verb!linenoplace! and \verb!convindent! are usually negative,
  and should be tenths of cm (if you are using cms for your lengths);
\item \verb!setarrowplace! is usually a positive integer (whole number);
\item the number in [ ] after \verb!\begin{convtran}! (if used) is
  usually positive but less than 1cm.
\end{itemize}

To adjust (increase) the right margin, use
\verb!\setconvrightmargin{length}! in the preamble.

Footnotes are also possible to some degree, by way of this kind of
coding:\footnote{Note however that the same mechanism for highlighting
  lines (i.e. $\backslash$\texttt{marginpar}) doesn't seem to be possible;
  likewise, line numbers disappear.
    \begin{convtran}[2cm]
    \setconvindent{1cm}
    \turn{Bob:} Hello Jim.  
    \turn{Jim:} Hello Bob.
    \end{convtran}}
\begin{verbatim}
\footnote{\begin{convtran}[2cm]
          \setconvindent{1cm}
          \turn{Bob:} Hello Jim.  
          \turn{Jim:} Hello Bob.
          \end{convtran}}
\end{verbatim}

\section{Starting line numbering at values other than 1}

If you want line numbering to start at something other than 1, use
\verb!\linenumbers[xx]!.  For instance, this
\begin{verbatim}
\begin{convtran}
  \linenumbers[11]
  \turn{Bob:} Hello Jim. \hl
  \turn{Jim:} Hello Bob.
\end{convtran}
\end{verbatim}
produces
\begin{convtran}
  \linenumbers[11]
  \turn{Bob:} Hello Jim. \hl
  \turn{Jim:} Hello Bob.
\end{convtran}

If you want to continue numbering from where an earlier excerpt ended,
use \verb!\convtocont! and \verb!\convcont!.  This:
\begin{verbatim}
\begin{convtran}
  \turn{A:} hello
  \turn{B:} hello
\end{convtran}\convtocont
Here is a continuation from where we left off:
\begin{convtran}\convcont
  \turn{A:} hello
  \turn{B:} hello
\end{convtran}
\end{verbatim}
produces this:
\begin{convtran}
  \turn{A:} hello
  \turn{B:} hello
\end{convtran}\convtocont
Here is a continuation from where we left off:
\begin{convtran}\convcont
  \turn{A:} hello
  \turn{B:} hello
\end{convtran}
Note that \verb!\convtocont! must go after \verb!\end{convtran}!: if
you put it before, your next line numbers will start with the same
number as the last line. The command \verb!\convcont! can even be used
after intervening examples, so long as there hasn't been an
intervening \verb!\convtocont! command.

You can remember where you were up to in more than one excerpt by
creating new counters and referring to those e.g. 
\begin{verbatim}
\newcounter{endofconvTwo}
\newcommand{\convtocontTwo}{\setcounter{endofconvTwo}{\thelinenumber}}
\newcommand{\convcontTwo}{\linenumbers[\theendofconvTwo]}
\end{verbatim}
You can then use \verb!\convcontTwo! to recall the new counter, or
\verb!\convcont! to recall the original one.

\section{Using convtran in a minipage}

(Code provided by Thomas Deacon.)  Line numbering can be preserved by
using the environment \verb!convtran*! and the commands
\verb!\turnmini! and \verb!\turnmini*! e.g.
\begin{verbatim}
\begin{minipage}{0.4\textwidth}
  \raggedleft
  \begin{convtran*}    
    \turnmini{A:} is this a question
    \turnmini{B:} is this a response
    \turnmini{A:} yes
  \end{convtran*}
\end{minipage}
\end{verbatim}
produces
\begin{minipage}{0.4\textwidth}
  \raggedleft
  \begin{convtran*}    
    \turnmini{A:} is this a question 
    \turnmini{B:} is this a response
    \turnmini{A:} yes
  \end{convtran*}
\end{minipage}

A minipage also makes it possible to put transcriptions, with line numbers, into beamer slides e.g.:
\begin{verbatim}
\documentclass{beamer}
\usepackage{convtran}
\begin{document}
\begin{frame}
  \begin{minipage}{\linewidth}
    \begin{convtran*}[1.2cm]
      \setlinenoplace{-0.3cm}
      \setconvindent{0.8cm}
      \turnmini{A:} when are you going to uh Tenerife? 
      \turnmini{B:} well I'm going on the February eighth and and
         then I'll probably be there like two weeks.
    \end{convtran*}
  \end{minipage}
\end{frame}
\end{document}
\end{verbatim}

\section{Other commands}

A range of other commands are provided:
\begin{description}
\item{\verb!\turn{label}{text}!} for numbered turns at talk
\item{\verb!\turn*{label}{text}!} for unnumbered turns at talk
\item{\verb!\phon{label}{text}!} for unnumbered phonetic details
\item{\verb!\info{label}{text}!} for unnumbered information
\item{\verb!\glos{label}{text}!} for unnumbered gloss lines
\item{\verb!\tran{label}{text}!} for unnumbered translation lines
\end{description}
Some symbols are also provided:
\begin{description}
\item{\verb!\high!} up arrow
\item{\verb!\low!} down arrow
\item{\verb!\q!} raised circle
\item{\verb!\ul{text}!} underlining
\item{\verb!\hl!} arrow in the margin
\item{\verb!\Hl!} double arrow in the margin
\item{\verb!\hllab{XX}{foo}!} labelled arrow in the margin, where
  \verb!XX! is however far you want to bump along the arrow in order
  to make space for the label, and \verb!foo! is the label
\item{\verb!\Hllab{XX}{foo}!} as previous, but with a double arrow
\item{\verb!\margmark{XX}!} arbitrary margin marks (XX in this case)
\end{description}

The package can be loaded with the options
\verb![single/onehalf/double]! to adjust the line spacing.  Default
(i.e. where no option is loaded) is single.  So, if you wanted double
line spacing, in your preamble use \verb!\usepackage[double]{convtran}!.

\section{Referring to figures}

You might want to refer to figures, e.g. screenshots.  For this use
the \verb!\figlab! command.
\begin{verbatim}
\begin{convtran}
  \turn{A:} one two three four
  \figlabs{}  \figlab{1}~~~~~~~~~\figlab{2}~~~~~~~~
              \figlab{3}~~~~~~\figlab{4}\vspace*{-5pt}%
  \turn{B:} this is a test of some labels
\end{convtran}
\end{verbatim}
produces
\begin{convtran}
  \turn{A:} one two three four
  \figlabs{}  \figlab{1}~~~~~~~~~\figlab{2}~~~~~~~~
              \figlab{3}~~~~~~\figlab{4}\vspace*{-5pt}%
  \turn{B:} this is a test of some labels
\end{convtran}
The \verb!\vspace*{-5pt}! command isn't essential, but it does put the
arrow a bit closer to the text.  Note that you can use the
\verb!\ref{}! command inside \verb!\figlab! to refer to a figure.

\section{Known problems}

\begin{itemize}
\item Using \verb![! at the start of a turn is fatal: use \verb!{[}!.
\item Using \verb!\hl!, \verb!\lab! \verb!\labhl!, \verb!\textlabhl!
  at the start of a turn (ie. before any text) is fatal.
\item The package uses \verb!\marginpar! to highlight lines with an
  arrow in the margin: this would cause problems for anyone using
  convtran in a document with \verb!\marginpar! used for other things
  e.g. users of \verb!classicthesis! and \verb!\todonotes!; if there
  are problems then the \verb!marginnote! package might help.
\item \verb!convtran! and \verb!rotating! clash as both define
  \verb!turn!.  Try this as a workaround:
\begin{verbatim}
  \usepackage{rotating}
  \let\turn\relax
  \usepackage{convtran}
\end{verbatim}
\item \verb!\hl! does not work when using \verb!\convtran*! inside a
  minipage.  The package
  \href{https://www.ctan.org/pkg/minipage-marginpar}{minipage-marginpar}
  might hold out some hope here.
\item The \href{https://www.ctan.org/pkg/acmart}{acmart} class doesn't
  seem to respect any repeated manual spaces (\verb!~~!, \verb!~~~!
  etc.) This applies in all contexts, not just within the
  \verb!convtran! environment. There are (at least) two ways around
  this. Where you are using for example \verb!~~~~~~! you could
  \begin{enumerate}
  \item Use \verb!\hphantom{xxxxxx}!, i.e. replace each \verb!~! with
    \verb!x!. \verb!\hphantom! inserts an amount of space equal to the
    width of the text within the command (good when using a monospaced
    font).
  \item Use e.g. \verb!\hspace*{1cm}!. \verb!\hspace!  inserts an
    amount of space equal to the width specified in it (good when
    creating transcriptions with variable space fonts).
  \end{enumerate}
\item If you use the \verb!\linelabel! or \verb!\lab! commands in a
  minipage (even if you don't refer to those lines), references later
  in the document seem to get messed up.
\item \verb!gb4e.sty! has to be loaded after \verb!convtran!.
\item \verb!convtran! loads \verb!lineno!. Loading \verb!csquotes!
  before \verb!convtran! may lead to the warning message
  \verb!Command \@parboxrestore has changed.!. See
  \href{https://tex.stackexchange.com/a/447159}{https://tex.stackexchange.com/a/447159}.

\end{itemize}

\end{document}

%%% Local Variables:
%%% mode: latex
%%% TeX-master: t
%%% End:
